% arara: pdflatex
% arara: pdflatex
% --------------------------------------------------------------------------
% the TRANSLATIONS package
% 
%   a simple translator
% 
% --------------------------------------------------------------------------
% Clemens Niederberger
% Web:    https://bitbucket.org/cgnieder/exsheets/
% E-Mail: contact@mychemistry.eu
% --------------------------------------------------------------------------
% Copyright 2011-2013 Clemens Niederberger
% 
% This work may be distributed and/or modified under the
% conditions of the LaTeX Project Public License, either version 1.3
% of this license or (at your option) any later version.
% The latest version of this license is in
%   http://www.latex-project.org/lppl.txt
% and version 1.3 or later is part of all distributions of LaTeX
% version 2005/12/01 or later.
% 
% This work has the LPPL maintenance status `maintained'.
% 
% The Current Maintainer of this work is Clemens Niederberger.
% --------------------------------------------------------------------------
% The translations package is part of the exsheets bundle
% --------------------------------------------------------------------------
% If you have any ideas, questions, suggestions or bugs to report, please
% feel free to contact me.
% --------------------------------------------------------------------------
%
% if you want to compile this documentation you'll need the document class
% `cnpkgdoc' which you can get here:
%    https://bitbucket.org/cgnieder/cnpkgdoc/
% the class is licensed LPPL 1.3 or later
%
% use `pdflatex' for compilation
%
\PassOptionsToPackage{supstfm=libertinesups}{superiors}
\documentclass[DIV9,toc=index,toc=bib,numbers=noendperiod]{cnpkgdoc}
% ----------------------------------------------------------------------------
% document layout and typographic features
\docsetup{
  pkg      = {translations} ,
  url      = {https://bitbucket.org/cgnieder/exsheets/} ,
  subtitle = {{\small part of the \href{exsheets_en.pdf}{\ExSheets} bundle}\\[\baselineskip]a simple translator} ,
  language = en ,
  modules  = true ,
  code-box = {
    backgroundcolor  = gray!7!white ,
    skipbelow        = .6\baselineskip plus .5ex minus .5ex ,
    skipabove        = .6\baselineskip plus .5ex minus .5ex ,
    innerleftmargin  = 1.5em ,
    innerrightmargin = 2.5em
  } ,
  gobble   = 1
}
\usepackage{libertinehologopatch}
\cnpkgusecolorscheme{friendly}

\usepackage[biblatex]{embrac}[2012/06/29]
  \ChangeEmph{[}[,.02em]{]}[.055em,-.08em]
  \ChangeEmph{(}[-.01em,.04em]{)}[.04em,-.05em]
\usepackage{microtype}
\usepackage[multiple]{fnpct}

\renewcommand*\othersectionlevelsformat[3]{\textcolor{main}{#3\autodot}\enskip}
\renewcommand*\partformat{\textcolor{main}{\partname~\thepart\autodot}}

\pagestyle{headings}

\setcapindent{1.5em}
\setkomafont{caption}{\normalfont\footnotesize\sffamily}
\setkomafont{captionlabel}{\normalfont\footnotesize\sffamily\scshape}

\usepackage{booktabs,array,ragged2e}

% ----------------------------------------------------------------------------
% code examples
\addcmds{
  cuisine,
  DeclareDictTranslation, DeclareLanguage, DeclareLanguageAlias,
    DeclareLanguageDialect, DeclareTranslation, DeclareTranslationFallback,
  GetTranslation, GetTranslationFor,
  kitchen,
  LoadDictionary, LoadDictionaryFor,
  mypackage@title,
  NewTranslation,
  ProvideDictionaryFor,
  RenewTranslation,RequirePackage,
  SaveTranslation, SaveTranslationFor, selectlanguage
}

% ----------------------------------------------------------------------------
% other packages, bibliography, index
\usepackage{xcoffins,wasysym,enumitem,booktabs,siunitx}
\usepackage[accsupp]{acro}
\DeclareAcronym{id}{
  short     = id ,
  long      = Identifier ,
  format    = \scshape ,
  pdfstring = ID ,
  accsupp   = ID
}

\usepackage{filecontents}
\usepackage{imakeidx}
\begin{filecontents*}{\jobname.ist}
 heading_prefix "{\\bfseries "
 heading_suffix "\\hfil}\\nopagebreak\n"
 headings_flag  1
 delim_0 "\\dotfill\\hyperpage{"
 delim_1 "\\dotfill\\hyperpage{"
 delim_2 "\\dotfill\\hyperpage{"
 delim_r "}\\textendash\\hyperpage{"
 delim_t "}"
 suffix_2p "\\nohyperpage{\\,f.}"
 suffix_3p "\\nohyperpage{\\,ff.}"
\end{filecontents*}
\indexsetup{othercode=\footnotesize}
\makeindex[options={-s \jobname.ist},intoc,columns=3]

\usepackage{csquotes}
\usepackage[backend=biber,style=alphabetic]{biblatex}


% ----------------------------------------------------------------------------
% example definitions that have to be done in the preamble:
\DeclareTranslation{English}{Kueche}{kitchen}
\DeclareTranslation{German}{Kueche}{K\"uche}
\DeclareTranslation{Spanish}{Kueche}{cocina}
\DeclareTranslation{French}{Kueche}{cuisine}
\DeclareTranslation{English}{farbe}{color}
\DeclareTranslation{British}{farbe}{colour}

% ----------------------------------------------------------------------------
% custom commands
\newcommand*\ExSheets{{\scshape\textcolor{main}{ExSheets}}\xspace}
\newcommand*\Default[1]{%
  \hfill\llap
    {%
      \ifblank{#1}%
        {(initially~empty)}%
        {Default:~\code{#1}}%
    }%
  \newline
}
\newcommand*\required{\hfill\llap{required}\newline}
\newcommand*\optional{\hfill\llap{optional}\newline}

\newcommand*\nooption[2]{\item[\code{\textcolor{key}{#1} = #2}] \cnpkgdocarrow\ }

\newcommand*\unexpsign{$\RHD$}
\newcommand*\expsign{\textcolor{red}{$\rhd$}}
\renewcommand*\cnpkgdoctriangle{\unexpsign}
\newcommand*\expandable{%
  \def\cnpkgdoctriangle{%
    \expsign
    \gdef\cnpkgdoctriangle{\unexpsign}}}

\usepackage{marginnote}
\makeatletter
\newcommand*\sinceversion[1]{%
  \@bsphack
  \marginnote{%
    \footnotesize\sffamily\RaggedRight
    \textcolor{black!75}{Introduced in version~#1}}%
  \@esphack}
\newcommand*\changedversion[1]{%
  \@bsphack
  \marginnote{%
    \footnotesize\sffamily\RaggedRight
    \textcolor{black!75}{Changed in version~#1}}%
  \@esphack}
\makeatother

\pdfstringdefDisableCommands{\def\llap#1{#1\space}\def\bigstar{*}}

% ----------------------------------------------------------------------------
% title page
% \TitlePicture{%
%   \parbox{.8\linewidth}{%
%   }%
% }

\begin{document}

\section{Motivation}\secidx{Motivation}
This package provides means for package authors to have an easy interface for
internationalization of their packages.  The functionality of this package is
in many parts also covered by the package \paket*{translator} (part of the
\paket{beamer} bundle).  Internationalization is also possible with
\paket{babel} and it's \cmd*{addto}\cmd{captions<language>} mechanism or
\KOMAScript's \cmd{providecaptionname}.  However, I believe that \translations
is more flexible than all of these. Unlike \paket*{translator} it detects the
used (\paket{babel} or \paket{polyglossia}) language itself and provides
expandable retrieving of the translated key.  \translations also provides
support for language dialects which means package authors can distinguish
between British, Australian, Canadian and US English, say.
\secidx*{Motivation}

\section{License and Requirements}\label{sec:license}\secidx{Requirements}
\translations is placed under the terms of the \LaTeX{} Project Public License,
version 1.3 or later (\url{http://www.latex-project.org/lppl.txt}). It has the
status ``maintained.''

\translations requires the \paket{etoolbox} package.
\secidx*{Requirements}

\section{Usage}\secidx{Usage}
\subsection{Available Commands}
Below the commands provided by \translations are explained. The symbol \expsign{}
means that the command is expandable, \unexpsign{} means that it isn't.
\begin{beschreibung}
 \Befehl{DeclareLanguage}{<lang>}\newline
   Declare a language that can be used by \translations. If the language already
   exists it will be silently redefined. This command can only be used in the preamble.
 \Befehl{DeclareLanguageAlias}{<lang2>}\ma{<lang1>}\newline
   Declares \code{<lang2>} to be an alias of \code{<lang1>}. If \code{<lang1>}
   doesn't exist yet a warning will be raised and it will be defined. This command
   can only be used in the preamble.
 \Befehl{DeclareLanguageDialect}{<dialect>}\ma{<lang>}\newline
   Declares \code{<dialect>} to be a dialect of language \code{<lang>}. If a
   translation for \code{<dialect>} is provided it is used by the translation
   macros. If there is none the corresponding translation for \code{<lang>} is
   used instead.
 \Befehl{NewTranslation}{<lang>}\ma{<key>}\ma{<translation>}\newline
   Defines a translation of key \code{<key>} for the language \code{<lang>}.
   An error will be raised if a translation of \code{<key>} already exists.
   This command can only be used in the preamble.
 \Befehl{RenewTranslation}{<lang>}\ma{<key>}\ma{<translation>}\newline
   Redefines a translation of key \code{<key>} for the language \code{<lang>}.
   An error will be raised if no translation of \code{<key>} exists.
   This command can only be used in the preamble.
 \Befehl{DeclareTranslation}{<lang>}\ma{<key>}\ma{<translation>}\newline
   Defines a translation of key \code{<key>} for the language \code{<lang>}.
   No error will be raised if a translation of \code{<key>} already exists.
   This command can only be used in the preamble.
 \Befehl{DeclareTranslationFallback}{<key>}\ma{<fallback>}\newline
   Defines a fallback translation for key \code{<key>} that is used in case no
   translation of \code{<key>} for the currently active language has been provided.
   No error will be raised if a fallback for \code{<key>} already exists.
   This command can only be used in the preamble.
 \expandable\Befehl{GetTranslationFor}{<lang>}\ma{<key>}\newline
   Fetches and prints the translation of \code{<key>} for the language \code{<lang>}.
   This command is expandable.
 \expandable\Befehl{GetTranslation}{<key>}\newline
   Fetches and prints the translation of \code{<key>} for the currently active
   language (as for example set by \paket{babel}). This command is expandable.
 \Befehl{SaveTranslationFor}{<cmd>}\ma{<lang>}\ma{<key>}\newline
   Fetches and saves the translation of \code{<key>} for the language \code{<lang>}
   in the macro \code{<cmd>}.
 \Befehl{SaveTranslation}{<cmd>}\ma{<key>}\newline
   Fetches and saves the translation of \code{<key>} for the currently active
   language (as for example set by \paket{babel}) in the macro \code{<cmd>}.
 \Befehl{LoadDictionary}{<name>}\newline
   Loads a file named \code{<name>-<lang>.trsl} where \code{<lang>} corresponds
   to the lowercase name of the current language as defined with
   \cmd{DeclareLanguage}. This file should contain the translations for the
   specified language.
 \Befehl{LoadDictionaryFor}{<lang>}\ma{<name>}\newline
   Loads a file named \code{<name>-<lang>.trsl}.
 \Befehl{DeclareDictTranslation}{<key>}\ma{<translation>}\newline
   This command is to be used in a dictionary file and picks up the language
   of that file, see section~\ref{sec:dictionaries} for an example.
\end{beschreibung}
Quite a number of languages already are defined, either directly or via an alias.
So, before you define a language you should take a look at
section~\ref{sec:languages} if the language doesn't already exist.

\subsection{A Small Example}
This section demonstrates with two short examples how the macros are used.
The first example covers the basics: dlaring of translations and then
retrieving and typesetting them.

\begin{beispiel}
 % in the preamble:
 % \DeclareTranslation{English}{Kueche}{kitchen}
 % \DeclareTranslation{German}{Kueche}{K\"uche}
 % \DeclareTranslation{Spanish}{Kueche}{cocina}
 % \DeclareTranslation{French}{Kueche}{cuisine}
 
 \GetTranslation{Kueche}
 \SaveTranslation\kitchen{Kueche}
 \SaveTranslationFor\cuisine{french}{Kueche}

 \selectlanguage{ngerman}
 \GetTranslation{Kueche} \kitchen\ \GetTranslationFor{spanish}{Kueche}
 \cuisine
\end{beispiel}

The next example demonstrates the use of dialects and how they fall back to
the translation for the main language if no extra translation was declared:

\begin{beispiel}
 % in the preamble:
 % \DeclareTranslation{English}{farbe}{color}
 % \DeclareTranslation{British}{farbe}{colour}

 \GetTranslationFor{English}{farbe} \\
 \GetTranslationFor{British}{farbe} \\
 \GetTranslationFor{American}{farbe}
\end{beispiel}

\subsection{Usage in Packages}
A typical usage in a package would look as follows:
\begin{beispiel}[code only]
 \RequirePackage{translations}
 \DeclareTranslationFallback{mypackage-title}{Nice Title}
 \DeclareTranslation{English}{mypackage-title}{Nice Title}
 \DeclareTranslation{French}{mypackage-title}{Beau Titre}
 \DeclareTranslation{German}{mypackage-title}{Sch\"{o}ner Titel} 
 ...
 \def\mypackage@title{\GetTranslation{mypackage-title}}
\end{beispiel}

That is, a package defines some unique key for an expression and at least
defines a fallback translation.  Additionally translations for as many
languages as the author wants are defined.  A user then may add
\cmd{DeclareTranslation}\ma{<language>}\ma{<translation>} if they find their
translation missing.

\subsection{Dictionaries}\label{sec:dictionaries}
A typical dictionary file should look as follows:
\begin{beispiel}[code only]
 % this is file housing-german.trsl
 \ProvideDictionaryFor{German}{housing}[<version info>]
 \DeclareDictTranslation{kitchen (housing)}{K\"uche}
 \DeclareDictTranslation{bathroom (housing)}{Bad}
 \DeclareDictTranslation{living room (housing)}{Wohnzimmer}
 \DeclareDictTranslation{bedroom (housing)}{Schlafzimmer}
 ...
 \endinput
\end{beispiel}

The usage is similar to the one in a package: unique keys are given
translations, this time for the language the dictionary file is declared for
only.

\secidx*{Usage}

\section{Defined Languages}\label{sec:languages}\secidx{Languages}
\translations currently has these languages defined, ``fallback'' being a
dummy language used for fallback translations:

\begin{quote}
  \makeatletter\raggedright
  \def\comma{\gdef\comma{,\quad}}%
  \def\do#1{\comma #1}%
  \dolistloop\@trnslt@languages
\end{quote}

To every one of these languages at least one alias exists, the uppercase
variant.  This is due to the fact that it is common to write language names
uppercased.  The defined aliases are these (in parentheses the base language
name is given):

\begin{quote}
  \makeatletter\raggedright
  \def\comma{\gdef\comma{,\quad}}
  \def\aliases#1#2{\comma #1 (#2)}
  \def\do#1{\aliases#1}
  \dolistloop\@trnslt@aliases@pair
\end{quote}

\translations also defines a few dialects.  The language to which the dialect
belongs to is given in paretheses:

\begin{quote}
  \makeatletter\raggedright
  \def\comma{\gdef\comma{,\quad}}
  \def\aliases#1#2{\comma #1 (#2)}
  \def\do#1{\aliases#1}
  \dolistloop\@trnslt@dialects@pair
\end{quote}
\secidx*{Languages}

These languages should cover all languages which are currently covered by
\paket{babel} and \paket{polyglossia}.

\section{Implementation}\index{Implementation@\textbf{Implementation}}
In the following code the lines 1--30 have been omitted. They only repeat the
license statement which has already been mentioned in section~\ref{sec:license}.

\implementation[linerange={31-1000},firstnumber=31]

\indexprologue{\noindent Section titles are indicated \textbf{bold}, packages
\textsf{sans serif}, commands \code{\textbackslash\textcolor{code}{brown}}
 and options \textcolor{key}{\code{yellow}}.\par\bigskip}

\printindex

\end{document}
